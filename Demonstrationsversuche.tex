\documentclass[12pt, a4paper]{article}

\usepackage{ngerman}
\usepackage{amsmath}
\usepackage{amssymb}
\usepackage{graphicx}

\title{\textbf{Praktikumsprotokoll - Bestimmung der spezifischen Ladung $\frac{e}{m}$}}
\author{Finn Koller, Nikolas Ullmerich}
\date{15.11.2016}

\begin{document}

\maketitle
\newpage
\tableofcontents
\newpage

\section{Demonstrationsversuche}
\subsection{D.1.}
Zu Beginn wurde der Off-Set-Druck am Manometer geme"sen. Dieser betrug 35 Pa. Danach wurde jeweils anhand einer Scheibensonde und einer Rohrsonde bei 2600 U/min und 1600 U/min der Druck senkrecht und parallel zur Str"omungsrichtung gemessen. \\
\\
\begin{tabular}{|l||c||c|c|}
\hline
  &   &  1600 U/min & 2600 U/min \\ 
\hline
Scheibensonde & senkrecht & -4 Pa & -10 Pa \\ 
\hline
& parallel & 35 Pa & 106 Pa \\ 
\hline
Rohrsonde & senkrecht & -35 Pa & -75 Pa \\ 
\hline
& parallel & 35 Pa & 105 Pa \\ 
\hline

\end{tabular}
\\
\\
Parallel zur Str"omungsrichtung wird der Gesamtdruck gemessen und senkrecht zur Str"omungsrichtung der statische Druck. \\
Bei Vergleich der gemessenen Werte der Rohrsonde mit der, der Scheibensonde f"allt auf, dass die parallel zu Str"omungsrichtung gemessenen Werte fast identisch sind, im anderen Fall ist der Unterschied allerdings sehr groß.\\
\\
Aus den Messdaten l"asst sich folgen, dass die Scheibensonde zur Messung des statischen Druckes besser geeignet ist, da dieser möglichst nah am Off-Set-Druck liegen sollte. Zur Messung des Gesamtdruckes kann man anscheinend sowohl die Rohrsonde, als auch die Scheibensonde verwenden. \\
Da der dynamische Druck aus Kombination des Gesamtdruckes sowie des statischen Druckes gemessen werden kann, empfiehlt es sich dazu, mit der Scheibensonde senkrecht zu messen und mit der Rohrsonde parallel.
\\
\subsection{D.2. Das Venturirohr}
\\
In diesem Versuch soll der Verlauf des statischen Druck an f"unf Stellen in einem Venturirohr (siehe Abbildung) beobachtet werden. Au"serdem soll der Gesamtdruck des betrachtet werden und mit unseren Erwartungen verglichen werden. \\
\\
Erwartungen zu diesem Versuch: \\
An der Engstelle des Venturirohr wird der statische Druck am geringsten erwartet, der dynamische Druck sollte allerdings dort am gr"o"sten sein, da die Querschnittsfl"ache abnimmt. \\
Der Gesamtdruck über das Rohres sollte konstant sein und es wird erwartet, dass vor und nach der Engstelle die Messdaten eine gewisse Symmetrie aufweisen.  \\
\\
Bild von dir ? \\
\\
Es l"asst sich erkennen, dass an der Engstelle der geringste statische Druck voriliegt, dies passt zu den Erwartungen, da hier auch der gr"o"ste dynamische Druck herrscht. Desweiteren f"allt auf, dass zwar vor der Engstelle die gemessenen Werte vom Beginn des Rohres bis zur Engstellle an Druck abnehmen, diesen aber nicht in gleicher Weise nach der Engstelle wieder zunehmen, was sich daran erkennen l"asst, dass die erste und die letzte Messung nicht "ubereinstimmen. Der Gesamtdruck scheint also nicht ganz konstant zu sein, was vermutlich an der Luftreibung im Rohr liegt. Eine weitere m"ogliche Fehlerquelle ist, dass die Öffnungen der Rohrsonden, durch die Form des Venturirohrs nicht ganz senkrecht zur Str"omungsrichtung liegen.
\\
\subsection{D.2. Das aerodynamische Paradoxon}
\\
In diesem Versuch wurde zwischen zwei engeinander liegenden Kreisscheiben
Druckluft axial zentrisch eingestr"omt. \\
Es konnte beobachtet werden, dass sich die zwei Kreisscheiben, ab einer gewissen Geschwindigkeit der Drucklust aneinander dr"uckten. \\
Dies erkl"art sich dadurch, dass die Fläche zwischen den Scheiben nach au"sen hin gr"o"ser wird. Dadurch nimmt die Str"mungsgeschwindigkeit nach  au"sen hin ab. Demenstprechend nimmt der statische Druck zu. Ab einer bestimmten Geschwindigkeit ist der statische Druck in der Mitte der Scheibe im Verh"altnis zu Umgebungsdruck so gering, dass ein Unterdruck zustande kommt, der die Platten aneinander dr"ckt.



\end{document}